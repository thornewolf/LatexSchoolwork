% !TEX options=--shell-escape

\documentclass{article}

\usepackage{multicol} % Enables multicol env.
\usepackage{graphicx} % Enables including graphics.
\usepackage[export]{adjustbox} % Allows graphics to have a `center` property.
\usepackage[pdf]{graphviz}
\usepackage{mdframed}
\usepackage{parskip}
\usepackage{titlesec} % Enables custom title sections.

\newmdtheoremenv{theo}{Theorem}

\newcommand{\YearPath}{../../../LatexConfig} % Get year level path (i.e. Senior)
\newcommand{\SemesterPath}{../../LatexConfig} % Get semester level path (i.e. Semester 1).
\newcommand{\ClassPath}{../LatexConfig} % Get class level path (i.e. CS101).
\newcommand{\AssignmentTitle}{Portfolio 1}
\newcommand{\AssignmentSub}{~ ~}

\renewcommand\thesubsection{(\alph{subsection})}
\titleformat{\subsection}
{\normalfont\normalsize}{\thesubsection}{1em}{}

\renewcommand\thesubsubsection{\thesubsection.\roman{subsubsection}}
\titleformat{\subsubsection}
{\normalfont\normalsize}{\thesubsubsection}{1em}{}

\begin{document}

\newcommand{\Professor}{Dr. Thorne Wolf}
\newcommand{\Course}{CS 101.1}
\newcommand{\Professor}{Dr. Thorne Wolf}
\newcommand{\Course}{CS 101.1}
\begin{titlepage}
  \begin{center}
      \vspace*{1cm}

      \Large
      \textbf{\AssignmentTitle}

      \vspace{0.5cm}
      \large
      \AssignmentSub \\
      ~\\
      \normalsize \today

      \vspace{1.5cm}

      \large
      By: \\
      ~\\
      \normalsize
      \vspace{1ex}
      \ifdefined\StudentNames \StudentNames \else Thorne Wolfenbarger\fi
      \vfill

      \vspace{0.8cm}

      \large
      Submitted to: \\
      \Professor \\
      In Partial Fulfillment of the Requirements of \\
      \Course~-~\Semester\\

      \vspace{0.8cm}

      \includegraphics[width=0.2\textwidth,center]{\YearPath/univ}
      ~\\
      College of Engineering\\
      Embry-Riddle Aeronautical University\\
      Prescott, AZ\\

  \end{center}
\end{titlepage}


\section{Homework 1 Question 2}
\subsection{Create $K_5$}
Figure \ref{fig:K5_0} displays a complete graph with five verticies.
\begin{figure}[ht]
\centering
\includegraphics[width=0.7\linewidth]{POD2_Graph1}
\caption{A Complete Graph With 5 Vertices ($K_5$)}
\label{fig:K5_0}
\end{figure}

\subsection{How many edges do you have?}
From an inspection of the circuit we can see that there are a total of 10 edges in this graph.

\newpage
\subsection{Does this graph have an Hamilton circuit?}
In order determine whether this graph has a Hamilton circuit we can use Dirac's theorem, as defined in Theorem \ref{theo:POD2:dt}.

\begin{theo}[Dirac's Theorem]
If G is a simple graph with n vertices with n $\geq$ 3 such that the
degree of every vertex in G is at least n/2, then G has a Hamilton circuit.
\label{theo:POD2:dt}
\end{theo}

Counting the degree of each vertex we can generate Table \ref{tab:POD2deg}:

\begin{table}[ht]
\begin{center}
\begin{tabular}{l|l}
Node & Degree \\
\hline
a & 4 \\
b & 4 \\
c & 4 \\
d & 4 \\
e & 4 \\
\end{tabular}
\end{center}
\caption{Degree of Each Node Within $K_5$}
\label{tab:POD2deg}
\end{table}

It can be seen that the degree for each vertex $v \in G$ , $deg(v)=4$ which is larger than 2.5. (Half of the number of vertices in the graph.) Therefore, by Theorem \ref{theo:POD2:dt} we determine that a Hamilton circuit exists.

\subsection{How many hamilton circuits does $K_5$ have?}
By the nature of a complete graph, we know that each pair of nodes has an edge connecting them. That means that we can traverse to any new node given some arbitrary starting node. In order to count the number of hamilton circuits within $K_5$ we need to determine how many possible paths exist within $K_5$ that visit each node exactly once.

The only constraint existent within this problem is that we are constrained to never visiting a node more than once. Due to the connectivity of the graph, this constraint will never prevent any non-repeating node patterns.

To count the number of solutions to this problem we first look at how many valid starting nodes exist within $K_5$. Due to the conclusion that a Hamilton circuit exists and due to the symmetry existing within $K_5$ we can say that if a Hamilton circuit exists starting from any single node, a Hamilton circuit must exist starting from any other node, due to them all being entirely symmetric. That is, we have 5 valid starting nodes.

We follow this by determining how many valid second nodes exist given some starting node has been selected. Due to the symmetry of the graph, we only need to determine the number of valid second nodes when starting at a single first node. We will arbitrarily choose Node A as our starting node and investigate possible second nodes.

Node 1 is connected to 4 distinct other nodes, each of which is identical to the other. That means we can arbitrarily visit a second node and conclude the number of valid second nodes is 4.

Figure \ref{fig:K5_1} displays our current state.

\begin{figure}[ht]
\centering
\includegraphics[width=0.8\linewidth]{POD2_Graph2}
\caption{A Complete Graph With 1 Traversed Edge ($K_5$)}
\label{fig:K5_1}
\end{figure}

\newpage
From this second node we have 3 identical options for the third visited node. Therefore he number of valid third nodes is 3. We arbitrarily visit Node C and update our state as seen in Figure \ref{fig:K5_2}.

\begin{figure}[ht]
\centering
\includegraphics[width=0.8\linewidth]{POD2_Graph3}
\caption{A Complete Graph With 2 Traversed Edges ($K_5$)}
\label{fig:K5_2}
\end{figure}

A completely identical argument can be used for determining options for the fourth and fifth node; 2 and 1 respectively. We can then use the product rule as detailed in Theorem \ref{theo:ProductRule}.

\begin{theo}[The Product Rule]
Suppose that a procedure can be broken down into a sequence of
two tasks. If there are $n_1$ ways to do the first task and for each of these ways of doing the first
task, there are $n_2$ ways to do the second task, then there are $n_1n_2$ ways to do the procedure.
\label{theo:ProductRule}
\end{theo}

The result of using Theorem \ref{theo:ProductRule} (The Product Rule) on each of our arbitrary 1st, 2nd, 3rd, 4th, and 5th nodes is $5\cdot4\cdot3\cdot2\cdot1=120$. There are 120 unique ways to generate a Hamilton circuit from $K_5$.

\subsection{When n is odd, how many Hamilton circuits does $K_n$ have?}
We can generalize our argument from $K_5$ to $K_n$ by taking note that the prerequisites to calculate a number of Hamilton circuits do not vary between any odd $n \geq 3$. Namely the degree of every vertex in G and the symmetry within a complete graph.


For any complete graph with $n$ vertices, the degree of each vertex is $n-1$. We require that the degree of each vertex be at least $n/2$ so we can solve the following inequality to determine what values of $n$ satisfy this constraint.

\begin{equation}
n-1 \geq n/2
\end{equation}

From this equation we determine that $n \geq 2$. We now have determined that $n-1 \geq n/2 ~\forall~ n \geq 2$. Therefore a Hamilton circuit exists for all $K_n$ where $n \geq 3$. We follow this with the same symmetry argument that allowed us to reduce $K_5$ into $5!$. Using the continual symmetry argument we can conclude that the number of Hamilton circuits in any odd $K_n$ to be $n!$.

\subsection{Revisions}
\subsubsection{How many edges does $K_n$ have? Prove this.}
Creating the graph $K_n$ involves creating a graph with $n$ vertices where each vertex is adjacent to $n-1$ other vertices. Determining now many edges exist within $K_n$ is not as simple as $n(n-1)$ because that would mean we have counted the edge connecting $(v_a, v_b)$ as both $(v_a, v_b)$ and $(v_b, v_a)$. This fact does give us insight into the correct number of edges within $K_n$. By specifically only counting pairs of $(v_i, v_j)$ where $i < j$ we then know that we can never count any pairing $(v_j, v_i)$ because that would be equivilant to some pairing $(v_i, v_j)$ where $i \geq j$, a direct contradiction to our previous restriction on all $i,j$ pairings.

So, will create an edge on every pair of nodes $i,j | 1 \leq i < j \leq n$. From this relationship we can see that there are $j-1$ distinct $i$ values for each $j \leq n$. We can then count the number of $i,j$ pairings within this relationship by summing each pairing together as follows.

\begin{equation}
  \sum_{j=1}^{n} \sum_{i=1}^{j-1} 1
\end{equation}

By performing algrbraic manipulations on the summations we find the following.

\begin{center}
  \begin{tabular}{cc}
    $\sum_{j=1}^{n} \sum_{i=1}^{j-1} 1$ & $\iff$ \\
    $\sum_{j=1}^{n} j-1$ & $\iff$ \\
    $\sum_{j=0}^{n-1} j$ & $\iff$ \\
    $\sum_{j=1}^{n-1} j$
  \end{tabular}
\end{center}

Within the final step we can see that this relationship is able to be coerced into a formula that is representative of the num of the first $n-1$ natural numbers. According to https://en.wikipedia.org/wiki/1\_\%2B\_2\_\%2B\_3\_\%2B\_4\_\%2B\_\%E2\%8B\%AF, this sum is known to be represented in the closed from equation $S = \frac{n(n-1)}{2}$. Therefore the number of edges within $K_n$ is $\frac{n(n-1)}{2}$. We verify this formula works in the case when with $n=5 \Rightarrow \frac{5\cdot4}{2}=10$. This has shown directly the solution for the number of edges within any graph $K_n$.

\subsubsection{How many Hamilton circuits does $K_n$ have?}
We can generalize our argument from $K_5$ to $K_n$ using proof by induction. The above proof has demonstrated that there are $5!$ ways to construct a hamilton circuit on $K_5$. We can reduce any graph $K_n$ into $K_{n-1}$ by removing a single vertex and all edges incident to this vertex. Conveniently enough, removing this single vertex $v$ is equivilant to visiting the vertex, and then eliminating it from the list of valid vertices. As there are $n$ vertices within $K_n$ that means we have $n$ ways of removing a single vertex from the graph. After the removal of this vertex from the graph, we are left with $K_{n-1}$. and if we assume that we can correctly determine the number of Hamilton circuits within $K_{n-1}$, we will represent the number of Hamilton circuits within $K_n$ as $n\cdot HP(K_{n-1})$ where $HP$ denotes the function to return the number of malilton circuits within a graph.

As we have proven $HP(K_5)=120$, we can reduct any arbitrary $K_n$ into $K_5$ using this technique. We can see that the solution generalizes to $n!$ for any $K_n$ due to this reduction. Q.E.D.

\subsection{Revision Summary}
Within this problem I have increased the amount of rigor associated with the generalization from $K_5$ to $K_n$. This comes in the form of providing two proofs as requested in the modified problem statement. Within the original problem my submission was very close to being a proof so the overall difference between the original solution and the present solution is not massive, but I have demonstrated a bit more closely how we transition into the sum of the first $n$ natural numbers as well as how to implicitly used induction to ddetermine $n!$ within the Hamilton circuit count.

\newpage


\section{Homework 2 Question 4}
\subsection{Every computer science student needs a course in discrete structures}
\subsubsection{Input Variables}
\begin{center}
  \begin{tabular}{ccl}
    $s$ & := & A single student. \\ 
    $S$ & := & The set of all students. \\ 
  \end{tabular}
\end{center}

\subsubsection{Define Necessary Propositions}
\begin{center}
  \begin{tabular}{ccl}
    $P(x)$ & := & student $x$ is a computer science student. \\
    $Q(x)$ & := & student $x$ needs a course in discrete structures. \\ 
  \end{tabular}
\end{center}

\subsubsection{Multivariate Quantifiers and Predicates}
\begin{center}
  $ \forall s \in S~P(s) \rightarrow Q(s)$
\end{center}


\subsection{Every student in this class has taken at least one programming course}

\subsubsection{Input Variables}
\begin{center}
  \begin{tabular}{ccl}
    $s$ & := & A student. \\
    $S$ & := & The set of all students. \\
  \end{tabular}
\end{center}

\subsubsection{Define Necessary Propositions}
\begin{center}
  \begin{tabular}{ccl}
    $P(x)$ & := & Student $x$ is in this class. \\
    $Q(x)$ & := & Student $x$ has taken at least one programming course.
  \end{tabular}
\end{center}


\subsubsection{Multivariate Quantifiers and Predicates}
\begin{center}
  $ \forall s \in S~P(s) \rightarrow Q(s) $
\end{center}


\subsection{There is a student in this class who has been in every room of at least one of the buildings on campus}

\subsubsection{Input Variables}
\begin{center}
  \begin{tabular}{ccl}
    $s$ & := & A single student. \\
    $S$ & := & The set of all students. \\
    $b$ & := & A single building. \\
    $B$ & := & The set of all buildings on campus. \\
  \end{tabular}
\end{center}

\subsubsection{Define Necessary Propositions}
\begin{center}
  \begin{tabular}{ccl}
    $P(x)$ & := & Student $x$ is a student in this class. \\
    $Q(x,y)$ & := & Student $x$ has been in every room of building $y$. \\
  \end{tabular}
\end{center}

\subsubsection{Multivariate Quantifiers and Predicates}
\begin{center}
  $ \exists s \in S~\exists b \in B~(P(s) \land Q(s, b)) $
\end{center}

\subsection{Revisions within this problem / summarized changes}
This problem was for the most part correct. Revisions that needed to be made included appropriately passing my ``element" variable to each predicate. Previously I had called $P(x)$ despite the only variable within the context of my quantified predicates being $s$. This has been corrected within the solution above.

The revisions and summary sections have been combined because of the relatively minor amount of revisions necessary to improve the quality of the submission. All changes were single characters, so they have been included in the solution.

\end{document}
