\subsection{Determine whether a tautology.}
In order to determine whether $(\lnot p \wedge (p \rightarrow q)) \rightarrow \lnot q$ is a tautology we can generate a truth table that progressively builds up each part of the expression until we can easily evaluate the truth table for the entire expression.


\begin{figure}[ht]
\begin{center}
\begin{tabular}{|c|c|c|c|c|c|c|}
    \hline
    p & q & $\lnot p$ & $p \rightarrow q$ & $\lnot p \wedge (p \rightarrow q)$ & $\lnot q$ & $(\lnot p \wedge (p \rightarrow q)) \rightarrow \lnot q$ \\
    \hline
    T & T & F & T & F & F & T \\
    \hline
    T & F & F & F & F & T & T \\
    \hline
    F & T & T & T & T & F & F \\
    \hline
    F & F & T & T & T & T & T \\
    \hline
\end{tabular}
\end{center}
\caption{Truth table verifying whether the above statement is a tautology.}
\end{figure}

From the above truth table we can see that the rightmost column contains at least one ``False'' value. Since the rightmost column represents the truthiness values associated with the entire expression, we determine that this is not a tautology.

\newpage
\subsection{Example of a statement which is a contradiction}
\begin{center}
$p \lor (q \wedge p) \wedge \lnot ((w \wedge p) \lor p) \wedge (x \leftrightarrow x)$
\end{center}

The truth table containing these terms is quite large, so an excel spreadsheet verifying that this expression is a contradiction is provided below.


\begin{figure}[ht]
\begin{center}
\includegraphics[width=300pt]{truthtable}
\end{center}
\caption{Truth table verifying that the above statement is a contradiction.}
\end{figure}
