\subsection{Create $K_5$}
Figure \ref{fig:K5_0} displays a complete graph with five verticies.
\begin{figure}[ht]
\centering
\includegraphics[width=0.7\linewidth]{POD2_Graph1}
\caption{A Complete Graph With 5 Vertices ($K_5$)}
\label{fig:K5_0}
\end{figure}

\subsection{How many edges do you have?}
From an inspection of the circuit we can see that there are a total of 10 edges in this graph.

\subsection{How many edges does $K_n$ have?}
We will create an edge on every pair of nodes $i,j | 1 \leq i < j \leq n$. From this relationship we can see that there are $j-1$ distinct $i$ values for each $j \leq n$. Counting the number of pairing generated here is eqivalalent to finding the sum of the first $n-1$ natural numbers. This sum is known to be represented in the closed from equation $S = \frac{n(n-1)}{2}$. Therefore the number of edges within $K_n$ is $\frac{n(n-1)}{2}$. We verify this formula works in the case when with $n=5 \Rightarrow \frac{5\cdot4}{2}=10$.

\newpage
\subsection{Does this graph have an Hamilton circuit?}
In order determine whether this graph has a Hamilton circuit we can use Dirac's theorem, as defined in Theorem \ref{theo:POD2:dt}.

\begin{theo}[Dirac's Theorem]
If G is a simple graph with n vertices with n $\geq$ 3 such that the
degree of every vertex in G is at least n/2, then G has a Hamilton circuit.
\label{theo:POD2:dt}
\end{theo}

Counting the degree of each vertex we can generate Table \ref{tab:POD2deg}:

\begin{table}[ht]
\begin{center}
\begin{tabular}{l|l}
Node & Degree \\
\hline
a & 4 \\
b & 4 \\
c & 4 \\
d & 4 \\
e & 4 \\
\end{tabular}
\end{center}
\caption{Degree of Each Node Within $K_5$}
\label{tab:POD2deg}
\end{table}

It can be seen that the degree for each vertex $v \in G$ , $deg(v)=4$ which is larger than 2.5. (Half of the number of vertices in the graph.) Therefore, by Theorem \ref{theo:POD2:dt} we determine that a Hamilton circuit exists.

\subsection{How many hamilton circuits does $K_5$ have?}
By the nature of a complete graph, we know that each pair of nodes has an edge connecting them. That means that we can traverse to any new node given some arbitrary starting node. In order to count the number of hamilton circuits within $K_5$ we need to determine how many possible paths exist within $K_5$ that visit each node exactly once.

The only constraint existent within this problem is that we are constrained to never visiting a node more than once. Due to the connectivity of the graph, this constraint will never prevent any non-repeating node patterns.

To count the number of solutions to this problem we first look at how many valid starting nodes exist within $K_5$. Due to the conclusion that a Hamilton circuit exists and due to the symmetry existing within $K_5$ we can say that if a Hamilton circuit exists starting from any single node, a Hamilton circuit must exist starting from any other node, due to them all being entirely symmetric. That is, we have 5 valid starting nodes.

We follow this by determining how many valid second nodes exist given some starting node has been selected. Due to the symmetry of the graph, we only need to determine the number of valid second nodes when starting at a single first node. We will arbitrarily choose Node A as our starting node and investigate possible second nodes.

Node 1 is connected to 4 distinct other nodes, each of which is identical to the other. That means we can arbitrarily visit a second node and conclude the number of valid second nodes is 4.

Figure \ref{fig:K5_1} displays our current state.

\begin{figure}[ht]
\centering
\includegraphics[width=0.8\linewidth]{POD2_Graph2}
\caption{A Complete Graph With 1 Traversed Edge ($K_5$)}
\label{fig:K5_1}
\end{figure}

\newpage
From this second node we have 3 identical options for the third visited node. Therefore he number of valid third nodes is 3. We arbitrarily visit Node C and update our state as seen in Figure \ref{fig:K5_2}.

\begin{figure}[ht]
\centering
\includegraphics[width=0.8\linewidth]{POD2_Graph3}
\caption{A Complete Graph With 2 Traversed Edges ($K_5$)}
\label{fig:K5_2}
\end{figure}

A completely identical argument can be used for determining options for the fourth and fifth node; 2 and 1 respectively. We can then use the product rule as detailed in Theorem \ref{theo:ProductRule}.

\begin{theo}[The Product Rule]
Suppose that a procedure can be broken down into a sequence of
two tasks. If there are $n_1$ ways to do the first task and for each of these ways of doing the first
task, there are $n_2$ ways to do the second task, then there are $n_1n_2$ ways to do the procedure.
\label{theo:ProductRule}
\end{theo}

The result of using Theorem \ref{theo:ProductRule} (The Product Rule) on each of our arbitrary 1st, 2nd, 3rd, 4th, and 5th nodes is $5\cdot4\cdot3\cdot2\cdot1=120$. There are 120 unique ways to generate a Hamilton circuit from $K_5$.

\subsection{When n is odd, how many Hamilton circuits does $K_n$ have?}
We can generalize our argument from $K_5$ to $K_n$ by taking note that the prerequisites to calculate a number of Hamilton circuits do not vary between any odd $n \geq 3$. Namely the degree of every vertex in G and the symmetry within a complete graph.


For any complete graph with $n$ vertices, the degree of each vertex is $n-1$. We require that the degree of each vertex be at least $n/2$ so we can solve the following inequality to determine what values of $n$ satisfy this constraint.

\begin{equation}
n-1 \geq n/2
\end{equation}

From this equation we determine that $n \geq 2$. We now have determined that $n-1 \geq n/2 ~\forall~ n \geq 2$. Therefore a Hamilton circuit exists for all $K_n$ where $n \geq 3$. We follow this with the same symmetry argument that allowed us to reduce $K_5$ into $5!$. Using the continual symmetry argument we can conclude that the number of Hamilton circuits in any odd $K_n$ to be $n!$.