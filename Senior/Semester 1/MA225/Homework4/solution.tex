% !TEX options=--shell-escape

\documentclass{article}

\usepackage{multicol} % Enables multicol env.
\usepackage{graphicx} % Enables including graphics.
\usepackage[export]{adjustbox} % Allows graphics to have a `center` property.
\usepackage[pdf]{graphviz}
\usepackage{mdframed}
\usepackage{parskip}
\usepackage{amssymb}
\usepackage{titlesec} % Enables custom title sections.
\usepackage{float}
\usepackage{listings}



\newmdtheoremenv{theo}{Theorem}
\newmdtheoremenv{fact}{Fact}
\newmdtheoremenv{definition}{Definition}
\newmdtheoremenv{lemma}{Lemma}

\renewcommand\thesection{Question \arabic{section}}
\titleformat{\subsection}
{\normalfont\normalsize}{\thesubsection}{1em}{}

\renewcommand\thesubsection{(\alph{subsection})}
\titleformat{\subsection}
{\normalfont\normalsize}{\thesubsection}{1em}{}

\renewcommand\thesubsubsection{\hspace{10px}\roman{subsubsection}.}
\titleformat{\subsubsection}
{\normalfont\normalsize}{\thesubsubsection}{1em}{}

\newcommand{\YearPath}{../../../LatexConfig} % Get year level path (i.e. Senior)
\newcommand{\SemesterPath}{../../LatexConfig} % Get semester level path (i.e. Semester 1).
\newcommand{\ClassPath}{../LatexConfig} % Get class level path (i.e. CS101).
\newcommand{\AssignmentTitle}{Homework 4}
\newcommand{\AssignmentSub}{~ ~}

\begin{document}

\newcommand{\Professor}{Dr. Thorne Wolf}
\newcommand{\Course}{CS 101.1}
\newcommand{\Professor}{Dr. Thorne Wolf}
\newcommand{\Course}{CS 101.1}
\begin{titlepage}
  \begin{center}
      \vspace*{1cm}

      \Large
      \textbf{\AssignmentTitle}

      \vspace{0.5cm}
      \large
      \AssignmentSub \\
      ~\\
      \normalsize \today

      \vspace{1.5cm}

      \large
      By: \\
      ~\\
      \normalsize
      \vspace{1ex}
      \ifdefined\StudentNames \StudentNames \else Thorne Wolfenbarger\fi
      \vfill

      \vspace{0.8cm}

      \large
      Submitted to: \\
      \Professor \\
      In Partial Fulfillment of the Requirements of \\
      \Course~-~\Semester\\

      \vspace{0.8cm}

      \includegraphics[width=0.2\textwidth,center]{\YearPath/univ}
      ~\\
      College of Engineering\\
      Embry-Riddle Aeronautical University\\
      Prescott, AZ\\

  \end{center}
\end{titlepage}


\section{} % 1
Define a recursive sequence \\
\begin{definition}
The $limit$ of a sequence $(a_n)$ as $n \rightarrow \infty$ exists and equals $L$ (denoted $\lim\limits_{n \rightarrow \infty} = L$) if and only if for any small number $\epsilon > 0$, there exists $N \in \mathbb{N}$ such that $|a_n - L| < \epsilon$ for all $n \geq N$.
\end{definition}
Define a sequence $a_n := 2(\frac{1}{3})^{n-1}$ for each $n \in \mathbb{Z}^+$

\subsection{What is the limit of $a_n$ as $n \rightarrow \infty$? Prove your answer using the definition above.}
We can demonstrate that the limit for the sequence $a_n$ is $0$ by demonstrating that there is some $N$ such that any $n \geq N$ the inequality $|a_n - L| < \epsilon$ for any arbitrary positive $\epsilon$. We will manipulate this inequality to see what value of $N$ could produce a valid result.

\begin{center}
  \begin{tabular}{ccc}
    $|a_n - L| < \epsilon$ & $\iff$ \\
    $\ln(a_n - 0) < \ln(\epsilon)$ & $\iff$ & $[1]$ \\
    $\ln(2(\frac{1}{3})^{n-1}) < \ln(\epsilon)$ & $\iff$ \\
    $\ln(2) + (n-1)\ln(\frac{1}{3}) < \ln(\epsilon)$ & $\iff$ \\
    $n - 1 > \frac{\ln(\epsilon) - \ln(2)}{\ln(\frac{1}{3})}$ & $\iff$ & $[2]$ \\
    $n > 1 + \ln(\epsilon/2) \cdot \ln(1/3)^{-1}$

  \end{tabular} \\
  \begin{tabular}{l}
  $[1]$ $\epsilon$ and an abolute value are both positive so this is valid. \\
  $[2]$ The direction of the inequality flips because we divide by a negative number ($ln(1/3)$).
  \end{tabular}
\end{center}

We can see that the above equation is satisfiable, as any arbitrarily defined $\epsilon$ will produce a value for $n$ to exceed. In order for this to be complete we need to know that $a_n$ is strictly decreasing as well. This can be seen because for any arbitrary value $n > 0$, $2(\frac{1}{3})^{n-1} > 2(\frac{1}{3})^{n+1-1}$. Therefore the limit exists and is 0.

\subsection{Let $S_k := \sum\limits_{n=1}^{k} a_n$ be the sum of the first $k$ terms in the sequence.}
\subsubsection{Compute $S_1, S_2, S_3, S_4, S_5$.}
\begin{center}
\begin{tabular}{l}
  $S_1 = \sum_{n=1}^1 = 2(\frac{1}{3})^{n-1} = 2(\frac{1}{3})^{1-1} = 2$ \\
  $S_2 = \sum_{n=1}^2 = 2(\frac{1}{3})^{n-1} = 2 + \frac{2}{3} = \frac{8}{3}$ \\
  $S_3 = \sum_{n=1}^3 = 2(\frac{1}{3})^{n-1} = \frac{8}{3} + \frac{2}{9} = \frac{26}{9}$ \\
  $S_4 = \sum_{n=1}^3 = 2(\frac{1}{3})^{n-1} = \frac{26}{9} + \frac{2}{27} = \frac{80}{27}$ \\
  $S_5 = \sum_{n=1}^3 = 2(\frac{1}{3})^{n-1} = \frac{80}{27} + \frac{2}{81} = \frac{242}{81}$ \\
\end{tabular}
\end{center}

\subsubsection{Does the sequence $S_k$ converge or diverge as $k \rightarrow \infty$?}
As $k \rightarrow \infty$, sequence $S_k$ converges. We know from part (a) that limit of $a_n$ (L) is 0. We know that the summation of a series that converges to 0 will also converge to some value. This value does not have to be 0 though.

\subsection{Let $S := \sum\limits_{n=1}^{\infty} a_n$. Find $S - \frac{1}{3}S$. Call this expression ($\star$)}
$S$ here is equivalent to $\lim\limits_{k \rightarrow \infty} \sum\limits_{n=1}^k 2(\frac{1}{3})^{n-1}$. This summation converges to 3. Therefore $S$ is equal to 3.

Alternatively, symbolically this is
\begin{center}
  \begin{tabular}{cc}
    $\sum\limits_{n=1}^\infty 2(\frac{1}{3})^{n-1} - \frac{1}{3}\sum\limits_{n=1}^\infty 2(\frac{1}{3})^{n-1}$ & $\iff$ \\
    $\sum\limits_{n=1}^\infty 2(\frac{1}{3})^{n-1} - \frac{2}{3}(\frac{1}{3})^{n-1}$ & $\iff$ \\
    $\sum\limits_{n=1}^\infty \frac{4}{3}(\frac{1}{3})^{n-1}$ & $\iff$ \\
    $\frac{4}{3} \sum\limits_{n=1}^\infty (\frac{1}{3})^{n-1}$ & $\iff$ \\
  \end{tabular}
\end{center}

This means $S - \frac{1}{3}S = 2 = (\star)$ = $\frac{4}{3} \sum\limits_{n=1}^\infty (\frac{1}{3})^{n-1}$.

\subsection{Solve the equation $S - \frac{1}{3}S = (\star)$ for $S$. Explain what this solution tells you.}
Solving this equation algebraically produces $S = \frac{(\star)}{2 / 3} = 2\sum\limits_{n=1}^\infty (\frac{1}{3})^{n-1} = 3$.

This relationship is demonstrating how $\sum\limits_{n=1}^\infty ar^{n-1} - r\sum\limits_{n=1}^\infty ar^{n-1} = 1$. This means that in general $S = \frac{a}{1-r}$. In this problem we have $a=2, r = \frac{1}{3}$ so $S = 3$.


\newpage
\section{} % 2
Create an interesting example of a relation on a set. Your set should have at least 5 elements, and your relation should be non-trivial.

Consider the set $S=\{2, 3, 6, 9, 12, 17\}$.\\
Consider the relation $R=\{(a,b) ~s.t.~ a$ shares a divisor with $b\}$

This means that the relation $R=$\\
$\{(2,6), (2,12), (3,6), (3,9), (3,12), (6,2), (6,3), (6,9), (6,12), (9,3), (9,6), (9,12),$\\
$(2,2), (3,3), (6,6), (9,9), (12,12), (17, 17)\}$

\subsection{Determine if the relation you defined on your chosen set is reflexive, symmetric, and/or transitive}
This relation is reflexive, symmetric, but not transitive. Reflexive because each element relates to itself, symmetric because $(a,b) \in R \rightarrow (b,a) \in R$, and not transitive because $(a,b) \in R \land (b,c) \in R \not \rightarrow (a,c) \in R$. An example of failing transitivity are the elements $\{(2,12), (12,9)\}$ but there is no $(2,9)$.

\subsection{Construct the relation matrix $M_R$ for the relation. Be sure to indicate which element of the set
represents which row of the matrix.}
\begin{figure}[H]
\centering
\begin{tabular}{|c|cccccc|}
\hline
Element & 2 & 3 & 6 & 9 & 12 & 17 \\
\hline
2 & 1 & 0 & 1 & 0 & 1 & 0 \\
3 & 0 & 1 & 1 & 1 & 1 & 0 \\
6 & 1 & 1 & 1 & 1 & 1 & 0 \\
9 & 0 & 1 & 1 & 1 & 1 & 0 \\
12 & 1 & 1 & 1 & 1 & 1 & 0 \\
17 & 0 & 0 & 0 & 0 & 0 & 1 \\
\hline
\end{tabular}
\end{figure}

\subsection{Construct a graph that corresponds to your relation. Be sure to indicate which element of the
set represents which vertex of the graph.}


\begin{figure}[H]
\centering
\includegraphics[width=4.5in]{RelationGraph}
\end{figure}

\subsection{Describe the relationship between your set relation, the relation matrix, the graph, and the adjacency matrix for the graph.}

The relation matrix $M_R$ is exactly the adjacency matrix for the resultant graph. A notable characteristic about the relation is that since it is symmetric we can represent the relation graph as an undirected graph. The reflexiveness of the relation is readily apparent since we can see the self-loop on every vertex within the graph. 

These three items, the relation, the relation matrix, and the resultant graph all communicate exactly the same information about this relation on the set $S$.

\newpage
\section{} % 3
\begin{definition}
The $limit$ of a sequence $(a_n)$ as $n \rightarrow \infty$ exists and equals $L$ (denoted $\lim\limits_{n \rightarrow \infty} = L$) if and only if for any small number $\epsilon > 0$, there exists $N \in \mathbb{N}$ such that $|a_n - L| < \epsilon$ for all $n \geq N$.
\end{definition}
Define a sequence $a_n := 2(\frac{1}{3})^{n-1}$ for each $n \in \mathbb{Z}^+$

\subsection{What is the limit of $a_n$ as $n \rightarrow \infty$? Prove your answer using the definition above.}
We can demonstrate that the limit for the sequence $a_n$ is $0$ by demonstrating that there is some $N$ such that any $n \geq N$ the inequality $|a_n - L| < \epsilon$ for any arbitrary positive $\epsilon$. We will manipulate this inequality to see what value of $N$ could produce a valid result.

\begin{center}
  \begin{tabular}{ccc}
    $|a_n - L| < \epsilon$ & $\iff$ \\
    $\ln(a_n - 0) < \ln(\epsilon)$ & $\iff$ & $[1]$ \\
    $\ln(2(\frac{1}{3})^{n-1}) < \ln(\epsilon)$ & $\iff$ \\
    $\ln(2) + (n-1)\ln(\frac{1}{3}) < \ln(\epsilon)$ & $\iff$ \\
    $n - 1 > \frac{\ln(\epsilon) - \ln(2)}{\ln(\frac{1}{3})}$ & $\iff$ & $[2]$ \\
    $n > 1 + \ln(\epsilon/2) \cdot \ln(1/3)^{-1}$

  \end{tabular} \\
  \begin{tabular}{l}
  $[1]$ $\epsilon$ and an abolute value are both positive so this is valid. \\
  $[2]$ The direction of the inequality flips because we divide by a negative number ($ln(1/3)$).
  \end{tabular}
\end{center}

We can see that the above equation is satisfiable, as any arbitrarily defined $\epsilon$ will produce a value for $n$ to exceed. In order for this to be complete we need to know that $a_n$ is strictly decreasing as well. This can be seen because for any arbitrary value $n > 0$, $2(\frac{1}{3})^{n-1} > 2(\frac{1}{3})^{n+1-1}$. Therefore the limit exists and is 0.

\subsection{Let $S_k := \sum\limits_{n=1}^{k} a_n$ be the sum of the first $k$ terms in the sequence.}
\subsubsection{Compute $S_1, S_2, S_3, S_4, S_5$.}
\begin{center}
\begin{tabular}{l}
  $S_1 = \sum_{n=1}^1 = 2(\frac{1}{3})^{n-1} = 2(\frac{1}{3})^{1-1} = 2$ \\
  $S_2 = \sum_{n=1}^2 = 2(\frac{1}{3})^{n-1} = 2 + \frac{2}{3} = \frac{8}{3}$ \\
  $S_3 = \sum_{n=1}^3 = 2(\frac{1}{3})^{n-1} = \frac{8}{3} + \frac{2}{9} = \frac{26}{9}$ \\
  $S_4 = \sum_{n=1}^3 = 2(\frac{1}{3})^{n-1} = \frac{26}{9} + \frac{2}{27} = \frac{80}{27}$ \\
  $S_5 = \sum_{n=1}^3 = 2(\frac{1}{3})^{n-1} = \frac{80}{27} + \frac{2}{81} = \frac{242}{81}$ \\
\end{tabular}
\end{center}

\subsubsection{Does the sequence $S_k$ converge or diverge as $k \rightarrow \infty$?}
As $k \rightarrow \infty$, sequence $S_k$ converges. We know from part (a) that limit of $a_n$ (L) is 0. We know that the summation of a series that converges to 0 will also converge to some value. This value does not have to be 0 though.

\subsection{Let $S := \sum\limits_{n=1}^{\infty} a_n$. Find $S - \frac{1}{3}S$. Call this expression ($\star$)}
$S$ here is equivalent to $\lim\limits_{k \rightarrow \infty} \sum\limits_{n=1}^k 2(\frac{1}{3})^{n-1}$. This summation converges to 3. Therefore $S$ is equal to 3.

Alternatively, symbolically this is
\begin{center}
  \begin{tabular}{cc}
    $\sum\limits_{n=1}^\infty 2(\frac{1}{3})^{n-1} - \frac{1}{3}\sum\limits_{n=1}^\infty 2(\frac{1}{3})^{n-1}$ & $\iff$ \\
    $\sum\limits_{n=1}^\infty 2(\frac{1}{3})^{n-1} - \frac{2}{3}(\frac{1}{3})^{n-1}$ & $\iff$ \\
    $\sum\limits_{n=1}^\infty \frac{4}{3}(\frac{1}{3})^{n-1}$ & $\iff$ \\
    $\frac{4}{3} \sum\limits_{n=1}^\infty (\frac{1}{3})^{n-1}$ & $\iff$ \\
  \end{tabular}
\end{center}

This means $S - \frac{1}{3}S = 2 = (\star)$ = $\frac{4}{3} \sum\limits_{n=1}^\infty (\frac{1}{3})^{n-1}$.

\subsection{Solve the equation $S - \frac{1}{3}S = (\star)$ for $S$. Explain what this solution tells you.}
Solving this equation algebraically produces $S = \frac{(\star)}{2 / 3} = 2\sum\limits_{n=1}^\infty (\frac{1}{3})^{n-1} = 3$.

This relationship is demonstrating how $\sum\limits_{n=1}^\infty ar^{n-1} - r\sum\limits_{n=1}^\infty ar^{n-1} = 1$. This means that in general $S = \frac{a}{1-r}$. In this problem we have $a=2, r = \frac{1}{3}$ so $S = 3$.


\newpage
\section{} % 4
Compute each of the following values, and show your work.
\subsection{($177$ mod $31\cdot 270$ mod $31$) mod 31}
\begin{center}
\begin{tabular}{cl}
$31\cdot 270 = 8370 > 177$ \\
177 mod 8370 = 177 & This is because 8370 is to large to divide 177 even once. \\
(177 mod 31) mod 31 = 22 mod 31 & The remainder after division is 22 \\
22 mod 31 = 22 & By the same reasoning as before. 31 does not divide 22
\end{tabular}
\end{center}

The answer is 22.

\subsection{($7^3$ mod 23)$^2$ mod 31}
\begin{center}
\begin{tabular}{clc}
($7^3$ mod 23)$^2$ mod 31 & & $\iff$ \\
$[(49 ~\rm{mod}~ 23)(7 ~\rm{mod}~ 23) ~\rm{mod}~ 23]^2$ mod 31 & By the modular multiplication identity & $\iff$ \\
$[(3)(7) ~\rm{mod}~ 23]^2 mod 31$ & Simplifying & $\iff$ \\
$21^2 ~\rm{mod}~ 31$ & Evaluating and simplifying & $\iff$ \\
$441 ~\rm{mod}~ 31 = 7$ & Evaluating
\end{tabular}
\end{center}

The answer is 7.

\subsection{$(99^2 ~\rm{mod}~ 32)^3 ~\rm{mod}~ 15$}

\begin{center}
\begin{tabular}{clc}
$(99^2 ~\rm{mod}~ 32)^3 ~\rm{mod}~ 15$ & & $\iff$ \\
$[(99 ~\rm{mod}~ 32)(99 ~\rm{mod}~ 32) ~\rm{mod}~ 32]^3 ~\rm{mod}~ 15$ & Modular multiplication rule & $\iff$ \\
$(3\cdot 3 ~\rm{mod}~ 32)^3 ~\rm{mod}~ 15$ & Evaluation & $\iff$ \\
$9^3 ~\rm{mod}~ 15$ & Evaluation & $\iff$ \\
$729 ~\rm{mod}~ 15 = 9$ & Evaluation
\end{tabular}
\end{center}

The answer is 9.

\subsection{Prove the following: if $a,b,m,n \in \mathbb{Z}, m \geq 1, n \geq 2$, and $a \equiv b ~\rm{mod}~ n$, then $a^m \equiv b^m ~\rm{mod}~ n$.}

\begin{center}
\begin{tabular}{p{0.6\linewidth}p{0.4\linewidth}c}
    $a,b,m,n \in \mathbb{Z}$ & Given as the domain for all used variables \\
    $m \geq 1, n \geq 2$ & Given \\
    $a \equiv b ~\rm{mod}~ n$ & Given \\
    $a^m = a \cdot a \cdot a\ldots \cdot a$ ($m$ times) & Definition of exponentiation \\
    $b^m = b \cdot b \cdot b\ldots \cdot b$ ($m$ times) & Definition of exponentiation \\
    $x^2 ~\rm{mod}~ q \iff (x ~\rm{mod}~ q)(x ~\rm{mod}~ q) ~\rm{mod}~ q$ & Law of modular multiplication for any arbitrary $x$. \\
    $a^m ~\rm{mod}~ n = (a ~\rm{mod}~ n)(a ~\rm{mod}~ n)\ldots (a ~\rm{mod}~ n) ~\rm{mod}~ n $ (m times) & Application of multiplication and exponentiation definitions \\
    $b^m ~\rm{mod}~ n = (b ~\rm{mod}~ n)(b ~\rm{mod}~ n)\ldots (b ~\rm{mod}~ n) ~\rm{mod}~ n $ (m times) & Application of multiplication and exponentiation definitions \\
    $b^m ~\rm{mod}~ n = (a ~\rm{mod}~ n)\ldots (a ~\rm{mod}~ n) ~\rm{mod}~ n $ (m times) & Since $a \equiv b ~\rm{mod}~ n$ \\
    $a^m \equiv b^m ~\rm{mod}~ n$ & Since we have established that both terms are equivalent to the multiplication expansion of $a^m ~\rm{mod}~ n$
\end{tabular}
\end{center}

Therefore, $a \equiv b ~\rm{mod}~ n \rightarrow a^m \equiv b^m ~\rm{mod}~ n$.

\newpage
\section{} % 5
\begin{definition}
The $limit$ of a sequence $(a_n)$ as $n \rightarrow \infty$ exists and equals $L$ (denoted $\lim\limits_{n \rightarrow \infty} = L$) if and only if for any small number $\epsilon > 0$, there exists $N \in \mathbb{N}$ such that $|a_n - L| < \epsilon$ for all $n \geq N$.
\end{definition}
Define a sequence $a_n := 2(\frac{1}{3})^{n-1}$ for each $n \in \mathbb{Z}^+$

\subsection{What is the limit of $a_n$ as $n \rightarrow \infty$? Prove your answer using the definition above.}
We can demonstrate that the limit for the sequence $a_n$ is $0$ by demonstrating that there is some $N$ such that any $n \geq N$ the inequality $|a_n - L| < \epsilon$ for any arbitrary positive $\epsilon$. We will manipulate this inequality to see what value of $N$ could produce a valid result.

\begin{center}
  \begin{tabular}{ccc}
    $|a_n - L| < \epsilon$ & $\iff$ \\
    $\ln(a_n - 0) < \ln(\epsilon)$ & $\iff$ & $[1]$ \\
    $\ln(2(\frac{1}{3})^{n-1}) < \ln(\epsilon)$ & $\iff$ \\
    $\ln(2) + (n-1)\ln(\frac{1}{3}) < \ln(\epsilon)$ & $\iff$ \\
    $n - 1 > \frac{\ln(\epsilon) - \ln(2)}{\ln(\frac{1}{3})}$ & $\iff$ & $[2]$ \\
    $n > 1 + \ln(\epsilon/2) \cdot \ln(1/3)^{-1}$

  \end{tabular} \\
  \begin{tabular}{l}
  $[1]$ $\epsilon$ and an abolute value are both positive so this is valid. \\
  $[2]$ The direction of the inequality flips because we divide by a negative number ($ln(1/3)$).
  \end{tabular}
\end{center}

We can see that the above equation is satisfiable, as any arbitrarily defined $\epsilon$ will produce a value for $n$ to exceed. In order for this to be complete we need to know that $a_n$ is strictly decreasing as well. This can be seen because for any arbitrary value $n > 0$, $2(\frac{1}{3})^{n-1} > 2(\frac{1}{3})^{n+1-1}$. Therefore the limit exists and is 0.

\subsection{Let $S_k := \sum\limits_{n=1}^{k} a_n$ be the sum of the first $k$ terms in the sequence.}
\subsubsection{Compute $S_1, S_2, S_3, S_4, S_5$.}
\begin{center}
\begin{tabular}{l}
  $S_1 = \sum_{n=1}^1 = 2(\frac{1}{3})^{n-1} = 2(\frac{1}{3})^{1-1} = 2$ \\
  $S_2 = \sum_{n=1}^2 = 2(\frac{1}{3})^{n-1} = 2 + \frac{2}{3} = \frac{8}{3}$ \\
  $S_3 = \sum_{n=1}^3 = 2(\frac{1}{3})^{n-1} = \frac{8}{3} + \frac{2}{9} = \frac{26}{9}$ \\
  $S_4 = \sum_{n=1}^3 = 2(\frac{1}{3})^{n-1} = \frac{26}{9} + \frac{2}{27} = \frac{80}{27}$ \\
  $S_5 = \sum_{n=1}^3 = 2(\frac{1}{3})^{n-1} = \frac{80}{27} + \frac{2}{81} = \frac{242}{81}$ \\
\end{tabular}
\end{center}

\subsubsection{Does the sequence $S_k$ converge or diverge as $k \rightarrow \infty$?}
As $k \rightarrow \infty$, sequence $S_k$ converges. We know from part (a) that limit of $a_n$ (L) is 0. We know that the summation of a series that converges to 0 will also converge to some value. This value does not have to be 0 though.

\subsection{Let $S := \sum\limits_{n=1}^{\infty} a_n$. Find $S - \frac{1}{3}S$. Call this expression ($\star$)}
$S$ here is equivalent to $\lim\limits_{k \rightarrow \infty} \sum\limits_{n=1}^k 2(\frac{1}{3})^{n-1}$. This summation converges to 3. Therefore $S$ is equal to 3.

Alternatively, symbolically this is
\begin{center}
  \begin{tabular}{cc}
    $\sum\limits_{n=1}^\infty 2(\frac{1}{3})^{n-1} - \frac{1}{3}\sum\limits_{n=1}^\infty 2(\frac{1}{3})^{n-1}$ & $\iff$ \\
    $\sum\limits_{n=1}^\infty 2(\frac{1}{3})^{n-1} - \frac{2}{3}(\frac{1}{3})^{n-1}$ & $\iff$ \\
    $\sum\limits_{n=1}^\infty \frac{4}{3}(\frac{1}{3})^{n-1}$ & $\iff$ \\
    $\frac{4}{3} \sum\limits_{n=1}^\infty (\frac{1}{3})^{n-1}$ & $\iff$ \\
  \end{tabular}
\end{center}

This means $S - \frac{1}{3}S = 2 = (\star)$ = $\frac{4}{3} \sum\limits_{n=1}^\infty (\frac{1}{3})^{n-1}$.

\subsection{Solve the equation $S - \frac{1}{3}S = (\star)$ for $S$. Explain what this solution tells you.}
Solving this equation algebraically produces $S = \frac{(\star)}{2 / 3} = 2\sum\limits_{n=1}^\infty (\frac{1}{3})^{n-1} = 3$.

This relationship is demonstrating how $\sum\limits_{n=1}^\infty ar^{n-1} - r\sum\limits_{n=1}^\infty ar^{n-1} = 1$. This means that in general $S = \frac{a}{1-r}$. In this problem we have $a=2, r = \frac{1}{3}$ so $S = 3$.


\newpage
\section{} % 6
POD 16: Below you will encrypt and decrypt messages using the RSA system.
\begin{definition}
The $limit$ of a sequence $(a_n)$ as $n \rightarrow \infty$ exists and equals $L$ (denoted $\lim\limits_{n \rightarrow \infty} = L$) if and only if for any small number $\epsilon > 0$, there exists $N \in \mathbb{N}$ such that $|a_n - L| < \epsilon$ for all $n \geq N$.
\end{definition}
Define a sequence $a_n := 2(\frac{1}{3})^{n-1}$ for each $n \in \mathbb{Z}^+$

\subsection{What is the limit of $a_n$ as $n \rightarrow \infty$? Prove your answer using the definition above.}
We can demonstrate that the limit for the sequence $a_n$ is $0$ by demonstrating that there is some $N$ such that any $n \geq N$ the inequality $|a_n - L| < \epsilon$ for any arbitrary positive $\epsilon$. We will manipulate this inequality to see what value of $N$ could produce a valid result.

\begin{center}
  \begin{tabular}{ccc}
    $|a_n - L| < \epsilon$ & $\iff$ \\
    $\ln(a_n - 0) < \ln(\epsilon)$ & $\iff$ & $[1]$ \\
    $\ln(2(\frac{1}{3})^{n-1}) < \ln(\epsilon)$ & $\iff$ \\
    $\ln(2) + (n-1)\ln(\frac{1}{3}) < \ln(\epsilon)$ & $\iff$ \\
    $n - 1 > \frac{\ln(\epsilon) - \ln(2)}{\ln(\frac{1}{3})}$ & $\iff$ & $[2]$ \\
    $n > 1 + \ln(\epsilon/2) \cdot \ln(1/3)^{-1}$

  \end{tabular} \\
  \begin{tabular}{l}
  $[1]$ $\epsilon$ and an abolute value are both positive so this is valid. \\
  $[2]$ The direction of the inequality flips because we divide by a negative number ($ln(1/3)$).
  \end{tabular}
\end{center}

We can see that the above equation is satisfiable, as any arbitrarily defined $\epsilon$ will produce a value for $n$ to exceed. In order for this to be complete we need to know that $a_n$ is strictly decreasing as well. This can be seen because for any arbitrary value $n > 0$, $2(\frac{1}{3})^{n-1} > 2(\frac{1}{3})^{n+1-1}$. Therefore the limit exists and is 0.

\subsection{Let $S_k := \sum\limits_{n=1}^{k} a_n$ be the sum of the first $k$ terms in the sequence.}
\subsubsection{Compute $S_1, S_2, S_3, S_4, S_5$.}
\begin{center}
\begin{tabular}{l}
  $S_1 = \sum_{n=1}^1 = 2(\frac{1}{3})^{n-1} = 2(\frac{1}{3})^{1-1} = 2$ \\
  $S_2 = \sum_{n=1}^2 = 2(\frac{1}{3})^{n-1} = 2 + \frac{2}{3} = \frac{8}{3}$ \\
  $S_3 = \sum_{n=1}^3 = 2(\frac{1}{3})^{n-1} = \frac{8}{3} + \frac{2}{9} = \frac{26}{9}$ \\
  $S_4 = \sum_{n=1}^3 = 2(\frac{1}{3})^{n-1} = \frac{26}{9} + \frac{2}{27} = \frac{80}{27}$ \\
  $S_5 = \sum_{n=1}^3 = 2(\frac{1}{3})^{n-1} = \frac{80}{27} + \frac{2}{81} = \frac{242}{81}$ \\
\end{tabular}
\end{center}

\subsubsection{Does the sequence $S_k$ converge or diverge as $k \rightarrow \infty$?}
As $k \rightarrow \infty$, sequence $S_k$ converges. We know from part (a) that limit of $a_n$ (L) is 0. We know that the summation of a series that converges to 0 will also converge to some value. This value does not have to be 0 though.

\subsection{Let $S := \sum\limits_{n=1}^{\infty} a_n$. Find $S - \frac{1}{3}S$. Call this expression ($\star$)}
$S$ here is equivalent to $\lim\limits_{k \rightarrow \infty} \sum\limits_{n=1}^k 2(\frac{1}{3})^{n-1}$. This summation converges to 3. Therefore $S$ is equal to 3.

Alternatively, symbolically this is
\begin{center}
  \begin{tabular}{cc}
    $\sum\limits_{n=1}^\infty 2(\frac{1}{3})^{n-1} - \frac{1}{3}\sum\limits_{n=1}^\infty 2(\frac{1}{3})^{n-1}$ & $\iff$ \\
    $\sum\limits_{n=1}^\infty 2(\frac{1}{3})^{n-1} - \frac{2}{3}(\frac{1}{3})^{n-1}$ & $\iff$ \\
    $\sum\limits_{n=1}^\infty \frac{4}{3}(\frac{1}{3})^{n-1}$ & $\iff$ \\
    $\frac{4}{3} \sum\limits_{n=1}^\infty (\frac{1}{3})^{n-1}$ & $\iff$ \\
  \end{tabular}
\end{center}

This means $S - \frac{1}{3}S = 2 = (\star)$ = $\frac{4}{3} \sum\limits_{n=1}^\infty (\frac{1}{3})^{n-1}$.

\subsection{Solve the equation $S - \frac{1}{3}S = (\star)$ for $S$. Explain what this solution tells you.}
Solving this equation algebraically produces $S = \frac{(\star)}{2 / 3} = 2\sum\limits_{n=1}^\infty (\frac{1}{3})^{n-1} = 3$.

This relationship is demonstrating how $\sum\limits_{n=1}^\infty ar^{n-1} - r\sum\limits_{n=1}^\infty ar^{n-1} = 1$. This means that in general $S = \frac{a}{1-r}$. In this problem we have $a=2, r = \frac{1}{3}$ so $S = 3$.


\end{document}
