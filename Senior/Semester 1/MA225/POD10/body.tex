Let S, T, R be sets.

\subsection{Prove that $(S - T) \cap (T - R) = \emptyset$}

We will perform a direct proof using set builder notation. Consider some set
S, T, R in some assumed larger universe U. Observe the following definitions for
(S - T) and (T - R) from the definition of the relative compiment of a set.

\begin{center}
\begin{tabular}{ll}
    $A = (S - T) = {x : (x \in S) \land (x \notin T)}$ & By the definition of relative compliment. \\
    $B = (T - R) = {x : (x \in T) \land (x \notin R)}$ & By the definition of relative compliment. \\
\end{tabular}
\end{center}

We have labeled these sets as A, B for notational convenience. By the definition
of set intersection the following relationship holds true.

\begin{center}
    $ A \cap B = {x : (x \in A) \land (x \in B)} $
\end{center}

By substituting in (S - T) and (T - R) in the place of A and B we can see the
following is true.

\begin{center}
    $(S - T) \cap (T - R) = {x : (x \in S) \land (x /\notin T) \land (x \in T) \land (x \notin R)} $
\end{center}

It can be trivially seen that $(x \notin T) \land (x \notin T)$ is a contradiction. There exists
no element x that is both within some set T and not within that very same set
T. Due to the existence of this contradiction we can see that the resultant set
must be equivlant to the empty set.

That is, $(S - T) \land (T - R) = \emptyset$. Q.E.D

\subsection{Given a set A in a larger universe U, let $\bar{A} = {x \in U : x \notin A}$. Suppose
S, T, R belong to some common universe U}

\newpage
\subsubsection{Draw the Venn diagram for the set}
\begin{figure}[ht]
\includegraphics[width=\linewidth]{venn.png}
\caption{Venn diagram describing this set}
\end{figure}

\newcommand{\thecaps}{\bar{S} \cap \bar{T} \cap \bar{R}}
\newcommand{\thecups}{\overline{S \cup T \cup R}}
\subsubsection{Prove that $ \thecaps = \thecups$}

We can prove this statement true by directly manupulating $x \in \thecups$ and demonstrating through a series of equivalency statements that this is equivalent to $x \in \thecaps$. We will assume that some arbitrary
element $x$ exists such that $x \in \thecups$. Observe:

\begin{center}
\begin{tabular}{ll}
    $x \in \thecups$ & $\iff$ \\
    $x \notin S \cup T \cup R$ & $\iff$ \\
    $x \notin S \land x \notin R \land x \notin T$ & $\star$ \\
\end{tabular}
\end{center}

The final step within these operations is crucial. This step is utilizing DeMorgan’s laws on sets to state that if some element is not within the union of a collection of sets, that element must not be in any of the constituent sets.

We can continue from this guarantee on x by using the fact that $x \notin A \equiv x \in \bar{A}$. Applying this to each individual set results in the following.

\begin{center}
\begin{tabular}{ll}
    $x \notin S \land x \notin R \land x \notin T$ & $\iff$ \\
    $x \in \bar{S} \land x \in \bar{R} \land x \in \bar{T}$ \\
\end{tabular}   
\end{center}

By applying the identity where $x \in A \land x \in B \equiv x \in A \cap B$ we can simplify the
above to the following.

\begin{center}
\begin{tabular}{ll}
    $x \in \bar{S} \land x \in \bar{R} \land x \in \bar{T}$ & $\iff$ \\
    $x \in \bar{S} \cap \bar{R} \cap \bar{T}$
\end{tabular}   
\end{center}

This has demonstrated that $x \in \thecups \iff x \in \thecaps$ due to the fact that all manipulations were biconditionals. Since $x$ is some arbitrary element, the previous biconditional means exactly that $ \thecaps = \thecups$. Q.E.D