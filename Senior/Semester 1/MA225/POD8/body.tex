\begin{raggedright}
Demonstrate that the following three statements are equivalent, where a and b are positive real numbers. i.e. $a,b > 0$ and $a,b \in \mathbb{R}$.
\end{raggedright}
\begin{center}
a is less than b \\
the average of a and b is greater than a \\
$\frac{b}{a} > 1$ \\
\end{center}

In order to prove that these three statements are equivalent, we will perform 2 distinct proofs and use the results from said proofs to establish biconditional dependence. Firstly, we will represent these three statements as three distinct propositions.

\begin{center}
\begin{tabular}{ccl}
  P & := & a is less than b \\
  Q & := & the average of a and b is greater than a \\
  R & := & $\frac{b}{a} > 1$ \\
\end{tabular}
\end{center}

The first proof will demonstrate that $P \iff Q$ is true, and the second proof will demonstrate $P \iff R$ is true.

\subsection{Demonstrate $P \iff Q$}
We will construct a direct proof by continually generating equivalent statements to convert $P$ to $Q$.

Let a and b be positive real numbers and suppose that a is less than b. Representing this in the form of an inequality we have $a < b$. Observe the following algebraic manipulations:

\begin{center}
  \begin{tabular}{cl}
    & $a < b$ \\
    $ \iff $ & $ a + a < b + a$ \\
    $ \iff $ & $ 2a < b + a$ \\
    $ \iff $ & $ a < \frac{b + a}{2}$ \\
    $ \iff $ & $ \frac{a + b}{2} > a$
  \end{tabular}
\end{center}

Each line provided is a symbolic manipulation that is equivalent to the line preceeding it. The above steps demonstrate that $a < b \equiv \frac{a + b}{2} > a$.

Note that $\frac{a + b}{2} > a$ is the matematical representation of "the average of a and b is greater than a." The above equivalency is precisely indicating that $P \iff Q$.

\subsection{Demonstrate $P \iff R$}
Let a and b be positive real numbers that a is less than b. We will construct a direct proof by generating a list of equivalent statements that demonstrate an equivalency between $P$ and $R$.

Representing this inequality mathematically we have $a < b$. Observe the following algebraic manipulation.

\begin{center}
  \begin{tabular}{cll}
    & $a < b$ \\
    $\iff$ & $1 < \frac{b}{a}$ & *
  \end{tabular}
\end{center}

Note that the operation on the step denoted with a * is only valid because we have established that both a and b are positive real numbers. This operation would not be valid if a were non-positive. We would either need to flip the inequality or consider that this operation is undefined for $a=0$.

The above trivially indicated that the inequality $a < b$ and be directly translated into the inequality $1 < \frac{b}{a}$. The latter inequality is $R$ exactly. Thus, we have demonstrated that $P \iff R$.

\subsection{Integrating previous proofs}
In the above text we have demonstrated both $P \iff Q$ and $P \iff R$. By the transitive biconditional theorem we can then use these results to conclude $Q \iff R$. We have now demonstrated the equivalency between all propositions $P,Q,R$.

Q.E.D

