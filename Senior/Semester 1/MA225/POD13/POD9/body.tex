
\subsection{Provide the full statement of at least one theorem, as it’s stated on the presenter’s slides.}
Every $C^*$-algebra looks like a closed $*$-subalgebra of $B(H)$ for some Hilbert space $H$.

\subsection{In the theorem statement above, what type of a proposition is it? Is it conditional? Does it have suppressed quantifiers? If so, rewrite the theorem statement showing these.}
It is a universal statement becaus ethe theorem states that ``Every $C^*$-algebra...'' an example of this is:

\begin{center}
  $c(x) = \{f : x \rightarrow \mathbb{C} : $f is continuous$\}$ where x is compact, Hausdorff...
\end{center}

I suppose that the theorem could be conditional because it has to be a $C^*$-algebra.

We can write this as
\begin{center}
  $\forall C^* algebra~\exists H \exists closed~*$-algebra$\in B(H)$.
\end{center}

\subsection{Write down a math term/word/notation that seems familiar to you (look out for words like injective, surjective, or notation like function notation).}
Let $(A(\mathbb{D}), \alpha)$ be a dynamical system. Then there exists a Mobius transformation of the form

\begin{center}
  $\tau(w) = \lambda (\frac{w-\mu}{1-\mu\bar w})$
\end{center}
such that $\alpha_1(f)(w) = f(\tau(w))$.

Within this I understand what it means for a system to be dynamical and for a transformation between spaces to exist.

\subsection{What do you understand this word/notation to mean?}
I understand this exerpt to mean there is some transform called a Mobius transform that exists as a function that accepts an argument $w$ and uses it to convert into something that is possibly called Mobius space.

\subsection{Using a resource (wikipedia is fine), provide a formal definition of the word or symbol as it was intended in the presentation.}
Here I look into the definition of a Mobius transformation.

In geometry and complex analysis, a Möbius transformation of the complex plane is a rational function of the form

\begin{center}
  $f(z) = \frac{az+b}{cz+d}$
\end{center}
of one complex variable $z$; here the coefficients $a,b,c,d$ are complex numbers satisfying $ab-bc \neq 0$.

"A Mobius transformation is always a bijective holomorphic function from the Riemann sphere to the Riemann sphere."

"These transformations preserve angles, map every straight line to a line or circle, and map every circle to a line or circle."

All information is probided from (Wikipedia, 2020). These exerpts provide the definition of a Mobius transformation as well as an invarient within the transformation.

\subsection{Write down a math term/word/notation that was totally unfamiliar to you.}
Irrational Rotation.

\subsection{Using a resource (wikipedia is fine), provide a formal definition of the word or symbol as it was intended in the presentation.}
In the mathematical theory of dynamicval systems, an irrational rotation is a map where $\theta$ is an irrational number. Under the identification of a circle with $R/z$ or with the interval $[0,1]$ with the boundary points glued together, this map becomes a rotation of a circle by a proportion $\theta$ of a full revolution.

\subsection{Ask a question that the presentation raised for you that you would like to know more about.}
Why isn't any orbit periodic? Within the same slide, Dr. Hamidi makes a statement "No orbit is periodic." What is a non-periodic orbit?
