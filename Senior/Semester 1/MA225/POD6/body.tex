\subsection{If the domain consists of all real numbers}
i. $\exists x(x^3 = -1)$ \\
There exists a number x such that $x^3=-1$. This statement is true for $x=-1$ because $(-1)^3=-1$.

ii. $\forall x(x^2+2 \geq 1)$ \\
Two more than the square of any arbitrary $x$ is greater than or equal to 1. \\
This is true because we can use algebra to re-represent the statement as $x^2 \geq -1$. The square of any real number $x$ is greater than 0. Since $0 > -1$, we can then conclude that $x^2 \geq -1$. Adding $2$ to both sides we can see that $x^2 + 2 \geq 1$, which is our original statement.

\subsection{If the domain consists of all functions on the real ine}
i. $\forall f(f(x) = f(y) \rightarrow x = y)$ \\
For every conceivable function $f$, if $f(x)$ is equal to $f(y)$, $x$ must be equal to $y$. \\
This is false because if we consider the function $f(q) = 1$ and use $x=5,y=6$ we satisfy $f(x) = f(x) \equiv f(5) = f(5) \equiv 1 = 1$ but fail to satisfy $x = y$ $(5 \neq 6)$.

ii. $\exists f(\forall y \exists x(f(x) = y))$ \\
There exists some function $f$ such that every possible value $y$ is within the range of $f$. \\
This is true because when $f(x) = x$, both the domain and range of $f$ are $(-\infty, \infty)$. A single value $x = y$ exists for all $y$ such that $f(x) = y.$
