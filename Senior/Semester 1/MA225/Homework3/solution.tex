% !TEX options=--shell-escape

\documentclass{article}

\usepackage{multicol} % Enables multicol env.
\usepackage{graphicx} % Enables including graphics.
\usepackage[export]{adjustbox} % Allows graphics to have a `center` property.
\usepackage[pdf]{graphviz}
\usepackage{mdframed}
\usepackage{parskip}
\usepackage{amssymb}
\usepackage{titlesec} % Enables custom title sections.
\usepackage{float}


\newmdtheoremenv{theo}{Theorem}
\newmdtheoremenv{definition}{Definition}
\newmdtheoremenv{lemma}{Lemma}

\renewcommand\thesubsection{(\alph{subsection})}
\titleformat{\subsection}
{\normalfont\normalsize}{\thesubsection}{1em}{}

\renewcommand\thesubsubsection{\hspace{10px}\roman{subsubsection}.}
\titleformat{\subsubsection}
{\normalfont\normalsize}{\thesubsubsection}{1em}{}

\newcommand{\YearPath}{../../../LatexConfig} % Get year level path (i.e. Senior)
\newcommand{\SemesterPath}{../../LatexConfig} % Get semester level path (i.e. Semester 1).
\newcommand{\ClassPath}{../LatexConfig} % Get class level path (i.e. CS101).
\newcommand{\AssignmentTitle}{Homework 3}
\newcommand{\AssignmentSub}{~ ~}

\begin{document}

\newcommand{\Professor}{Dr. Thorne Wolf}
\newcommand{\Course}{CS 101.1}
\newcommand{\Professor}{Dr. Thorne Wolf}
\newcommand{\Course}{CS 101.1}
\begin{titlepage}
  \begin{center}
      \vspace*{1cm}

      \Large
      \textbf{\AssignmentTitle}

      \vspace{0.5cm}
      \large
      \AssignmentSub \\
      ~\\
      \normalsize \today

      \vspace{1.5cm}

      \large
      By: \\
      ~\\
      \normalsize
      \vspace{1ex}
      \ifdefined\StudentNames \StudentNames \else Thorne Wolfenbarger\fi
      \vfill

      \vspace{0.8cm}

      \large
      Submitted to: \\
      \Professor \\
      In Partial Fulfillment of the Requirements of \\
      \Course~-~\Semester\\

      \vspace{0.8cm}

      \includegraphics[width=0.2\textwidth,center]{\YearPath/univ}
      ~\\
      College of Engineering\\
      Embry-Riddle Aeronautical University\\
      Prescott, AZ\\

  \end{center}
\end{titlepage}


\section{Question 1}
\begin{definition}
The $limit$ of a sequence $(a_n)$ as $n \rightarrow \infty$ exists and equals $L$ (denoted $\lim\limits_{n \rightarrow \infty} = L$) if and only if for any small number $\epsilon > 0$, there exists $N \in \mathbb{N}$ such that $|a_n - L| < \epsilon$ for all $n \geq N$.
\end{definition}
Define a sequence $a_n := 2(\frac{1}{3})^{n-1}$ for each $n \in \mathbb{Z}^+$

\subsection{What is the limit of $a_n$ as $n \rightarrow \infty$? Prove your answer using the definition above.}
We can demonstrate that the limit for the sequence $a_n$ is $0$ by demonstrating that there is some $N$ such that any $n \geq N$ the inequality $|a_n - L| < \epsilon$ for any arbitrary positive $\epsilon$. We will manipulate this inequality to see what value of $N$ could produce a valid result.

\begin{center}
  \begin{tabular}{ccc}
    $|a_n - L| < \epsilon$ & $\iff$ \\
    $\ln(a_n - 0) < \ln(\epsilon)$ & $\iff$ & $[1]$ \\
    $\ln(2(\frac{1}{3})^{n-1}) < \ln(\epsilon)$ & $\iff$ \\
    $\ln(2) + (n-1)\ln(\frac{1}{3}) < \ln(\epsilon)$ & $\iff$ \\
    $n - 1 > \frac{\ln(\epsilon) - \ln(2)}{\ln(\frac{1}{3})}$ & $\iff$ & $[2]$ \\
    $n > 1 + \ln(\epsilon/2) \cdot \ln(1/3)^{-1}$

  \end{tabular} \\
  \begin{tabular}{l}
  $[1]$ $\epsilon$ and an abolute value are both positive so this is valid. \\
  $[2]$ The direction of the inequality flips because we divide by a negative number ($ln(1/3)$).
  \end{tabular}
\end{center}

We can see that the above equation is satisfiable, as any arbitrarily defined $\epsilon$ will produce a value for $n$ to exceed. In order for this to be complete we need to know that $a_n$ is strictly decreasing as well. This can be seen because for any arbitrary value $n > 0$, $2(\frac{1}{3})^{n-1} > 2(\frac{1}{3})^{n+1-1}$. Therefore the limit exists and is 0.

\subsection{Let $S_k := \sum\limits_{n=1}^{k} a_n$ be the sum of the first $k$ terms in the sequence.}
\subsubsection{Compute $S_1, S_2, S_3, S_4, S_5$.}
\begin{center}
\begin{tabular}{l}
  $S_1 = \sum_{n=1}^1 = 2(\frac{1}{3})^{n-1} = 2(\frac{1}{3})^{1-1} = 2$ \\
  $S_2 = \sum_{n=1}^2 = 2(\frac{1}{3})^{n-1} = 2 + \frac{2}{3} = \frac{8}{3}$ \\
  $S_3 = \sum_{n=1}^3 = 2(\frac{1}{3})^{n-1} = \frac{8}{3} + \frac{2}{9} = \frac{26}{9}$ \\
  $S_4 = \sum_{n=1}^3 = 2(\frac{1}{3})^{n-1} = \frac{26}{9} + \frac{2}{27} = \frac{80}{27}$ \\
  $S_5 = \sum_{n=1}^3 = 2(\frac{1}{3})^{n-1} = \frac{80}{27} + \frac{2}{81} = \frac{242}{81}$ \\
\end{tabular}
\end{center}

\subsubsection{Does the sequence $S_k$ converge or diverge as $k \rightarrow \infty$?}
As $k \rightarrow \infty$, sequence $S_k$ converges. We know from part (a) that limit of $a_n$ (L) is 0. We know that the summation of a series that converges to 0 will also converge to some value. This value does not have to be 0 though.

\subsection{Let $S := \sum\limits_{n=1}^{\infty} a_n$. Find $S - \frac{1}{3}S$. Call this expression ($\star$)}
$S$ here is equivalent to $\lim\limits_{k \rightarrow \infty} \sum\limits_{n=1}^k 2(\frac{1}{3})^{n-1}$. This summation converges to 3. Therefore $S$ is equal to 3.

Alternatively, symbolically this is
\begin{center}
  \begin{tabular}{cc}
    $\sum\limits_{n=1}^\infty 2(\frac{1}{3})^{n-1} - \frac{1}{3}\sum\limits_{n=1}^\infty 2(\frac{1}{3})^{n-1}$ & $\iff$ \\
    $\sum\limits_{n=1}^\infty 2(\frac{1}{3})^{n-1} - \frac{2}{3}(\frac{1}{3})^{n-1}$ & $\iff$ \\
    $\sum\limits_{n=1}^\infty \frac{4}{3}(\frac{1}{3})^{n-1}$ & $\iff$ \\
    $\frac{4}{3} \sum\limits_{n=1}^\infty (\frac{1}{3})^{n-1}$ & $\iff$ \\
  \end{tabular}
\end{center}

This means $S - \frac{1}{3}S = 2 = (\star)$ = $\frac{4}{3} \sum\limits_{n=1}^\infty (\frac{1}{3})^{n-1}$.

\subsection{Solve the equation $S - \frac{1}{3}S = (\star)$ for $S$. Explain what this solution tells you.}
Solving this equation algebraically produces $S = \frac{(\star)}{2 / 3} = 2\sum\limits_{n=1}^\infty (\frac{1}{3})^{n-1} = 3$.

This relationship is demonstrating how $\sum\limits_{n=1}^\infty ar^{n-1} - r\sum\limits_{n=1}^\infty ar^{n-1} = 1$. This means that in general $S = \frac{a}{1-r}$. In this problem we have $a=2, r = \frac{1}{3}$ so $S = 3$.

\newpage

\section{Question 2}
Let $A = {1,3}$ and let $B = {1,2,3,4}$.
\subsection{Describe the set $A \times B$ using the roster method.}
$A \times B = \{(1,1), (1, 2), (1, 3), (1, 4), (3,1), (3, 2), (3, 3), (3, 4)\}$

\subsection{Describe the set $A \times B$ using the roster method.}
$B \times A = \{(1, 1), (1, 3), (2, 1), (2, 3), (3, 1), (3, 3), (4, 1), (4, 3), \}$

\subsection{Are the sets $A \times B$ and $B \times A$ equal? Why or why not?}
They are not equal sets. The cartesian product operator is not a communative operator. It can as the elements within the cartesian product of $A, B$ are ordered pairs, the order of $A, B$ within the operation matters. 

\subsection{Find $A \cup B$}
$A \cup B = \{1, 2, 3, 4\}$

\subsection{Find $A \cap B$}
$A \cap B = \{1, 3\}$

\subsection{Prove the following: Let $A$ and $B$ be two sets. Then $A \subseteq B$ if and only if $A \cup B = B$.}
To say that $A \subseteq B$ we are saying $\forall x \in A (x \in B)$. In set builder notation we can represent the set $A$ as $\{x : x \in A\}$. Since we know $\forall x \in A (x \in B)$ we can additionally constrain the previous set producing $\{x : x \in A\} = \{x : x \in A \land x \in B\}$. These statements are all equivalent, so we can represent this as $\{x : x \in A\} = \{x : x \in A \land x \in B\} \iff \forall x \in A (x \in B)$

Representing $A \cup B$ in set notation provides us with $\{x : x \in A \lor x \in B\}$. If $A \cup B = B$, that means $\{x : x \in A \lor x \in B\} = \{x : x \in B\}$ The only way this could be possible was if $\forall x \in A (x \in B)$ was true. That is, $\forall x \in A (x \in B) \iff \{x : x \in A \lor x \in B\} = \{x : x \in B\}$

We can see that both statements are equivalent to $\forall x \in A (x \in B)$.

\begin{center}
\begin{tabular}{c}
$\{x : x \in A\} = \{x : x \in A \land x \in B\} \iff \forall x \in A (x \in B)$ \\
$\forall x \in A (x \in B) \iff \{x : x \in A \lor x \in B\} = \{x : x \in B\}$\\
$\{x : x \in A\} = \{x : x \in A \land x \in B\} \iff \{x : x \in A \lor x \in B\} = \{x : x \in B\}$ \\
${A \subseteq B \iff A \cup B = B}$
\end{tabular}
\end{center}
Q.E.D
\newpage


\section{Question 3}
\begin{definition}
The $limit$ of a sequence $(a_n)$ as $n \rightarrow \infty$ exists and equals $L$ (denoted $\lim\limits_{n \rightarrow \infty} = L$) if and only if for any small number $\epsilon > 0$, there exists $N \in \mathbb{N}$ such that $|a_n - L| < \epsilon$ for all $n \geq N$.
\end{definition}
Define a sequence $a_n := 2(\frac{1}{3})^{n-1}$ for each $n \in \mathbb{Z}^+$

\subsection{What is the limit of $a_n$ as $n \rightarrow \infty$? Prove your answer using the definition above.}
We can demonstrate that the limit for the sequence $a_n$ is $0$ by demonstrating that there is some $N$ such that any $n \geq N$ the inequality $|a_n - L| < \epsilon$ for any arbitrary positive $\epsilon$. We will manipulate this inequality to see what value of $N$ could produce a valid result.

\begin{center}
  \begin{tabular}{ccc}
    $|a_n - L| < \epsilon$ & $\iff$ \\
    $\ln(a_n - 0) < \ln(\epsilon)$ & $\iff$ & $[1]$ \\
    $\ln(2(\frac{1}{3})^{n-1}) < \ln(\epsilon)$ & $\iff$ \\
    $\ln(2) + (n-1)\ln(\frac{1}{3}) < \ln(\epsilon)$ & $\iff$ \\
    $n - 1 > \frac{\ln(\epsilon) - \ln(2)}{\ln(\frac{1}{3})}$ & $\iff$ & $[2]$ \\
    $n > 1 + \ln(\epsilon/2) \cdot \ln(1/3)^{-1}$

  \end{tabular} \\
  \begin{tabular}{l}
  $[1]$ $\epsilon$ and an abolute value are both positive so this is valid. \\
  $[2]$ The direction of the inequality flips because we divide by a negative number ($ln(1/3)$).
  \end{tabular}
\end{center}

We can see that the above equation is satisfiable, as any arbitrarily defined $\epsilon$ will produce a value for $n$ to exceed. In order for this to be complete we need to know that $a_n$ is strictly decreasing as well. This can be seen because for any arbitrary value $n > 0$, $2(\frac{1}{3})^{n-1} > 2(\frac{1}{3})^{n+1-1}$. Therefore the limit exists and is 0.

\subsection{Let $S_k := \sum\limits_{n=1}^{k} a_n$ be the sum of the first $k$ terms in the sequence.}
\subsubsection{Compute $S_1, S_2, S_3, S_4, S_5$.}
\begin{center}
\begin{tabular}{l}
  $S_1 = \sum_{n=1}^1 = 2(\frac{1}{3})^{n-1} = 2(\frac{1}{3})^{1-1} = 2$ \\
  $S_2 = \sum_{n=1}^2 = 2(\frac{1}{3})^{n-1} = 2 + \frac{2}{3} = \frac{8}{3}$ \\
  $S_3 = \sum_{n=1}^3 = 2(\frac{1}{3})^{n-1} = \frac{8}{3} + \frac{2}{9} = \frac{26}{9}$ \\
  $S_4 = \sum_{n=1}^3 = 2(\frac{1}{3})^{n-1} = \frac{26}{9} + \frac{2}{27} = \frac{80}{27}$ \\
  $S_5 = \sum_{n=1}^3 = 2(\frac{1}{3})^{n-1} = \frac{80}{27} + \frac{2}{81} = \frac{242}{81}$ \\
\end{tabular}
\end{center}

\subsubsection{Does the sequence $S_k$ converge or diverge as $k \rightarrow \infty$?}
As $k \rightarrow \infty$, sequence $S_k$ converges. We know from part (a) that limit of $a_n$ (L) is 0. We know that the summation of a series that converges to 0 will also converge to some value. This value does not have to be 0 though.

\subsection{Let $S := \sum\limits_{n=1}^{\infty} a_n$. Find $S - \frac{1}{3}S$. Call this expression ($\star$)}
$S$ here is equivalent to $\lim\limits_{k \rightarrow \infty} \sum\limits_{n=1}^k 2(\frac{1}{3})^{n-1}$. This summation converges to 3. Therefore $S$ is equal to 3.

Alternatively, symbolically this is
\begin{center}
  \begin{tabular}{cc}
    $\sum\limits_{n=1}^\infty 2(\frac{1}{3})^{n-1} - \frac{1}{3}\sum\limits_{n=1}^\infty 2(\frac{1}{3})^{n-1}$ & $\iff$ \\
    $\sum\limits_{n=1}^\infty 2(\frac{1}{3})^{n-1} - \frac{2}{3}(\frac{1}{3})^{n-1}$ & $\iff$ \\
    $\sum\limits_{n=1}^\infty \frac{4}{3}(\frac{1}{3})^{n-1}$ & $\iff$ \\
    $\frac{4}{3} \sum\limits_{n=1}^\infty (\frac{1}{3})^{n-1}$ & $\iff$ \\
  \end{tabular}
\end{center}

This means $S - \frac{1}{3}S = 2 = (\star)$ = $\frac{4}{3} \sum\limits_{n=1}^\infty (\frac{1}{3})^{n-1}$.

\subsection{Solve the equation $S - \frac{1}{3}S = (\star)$ for $S$. Explain what this solution tells you.}
Solving this equation algebraically produces $S = \frac{(\star)}{2 / 3} = 2\sum\limits_{n=1}^\infty (\frac{1}{3})^{n-1} = 3$.

This relationship is demonstrating how $\sum\limits_{n=1}^\infty ar^{n-1} - r\sum\limits_{n=1}^\infty ar^{n-1} = 1$. This means that in general $S = \frac{a}{1-r}$. In this problem we have $a=2, r = \frac{1}{3}$ so $S = 3$.

\newpage

\section{Question 4}
\begin{definition}
The $limit$ of a sequence $(a_n)$ as $n \rightarrow \infty$ exists and equals $L$ (denoted $\lim\limits_{n \rightarrow \infty} = L$) if and only if for any small number $\epsilon > 0$, there exists $N \in \mathbb{N}$ such that $|a_n - L| < \epsilon$ for all $n \geq N$.
\end{definition}
Define a sequence $a_n := 2(\frac{1}{3})^{n-1}$ for each $n \in \mathbb{Z}^+$

\subsection{What is the limit of $a_n$ as $n \rightarrow \infty$? Prove your answer using the definition above.}
We can demonstrate that the limit for the sequence $a_n$ is $0$ by demonstrating that there is some $N$ such that any $n \geq N$ the inequality $|a_n - L| < \epsilon$ for any arbitrary positive $\epsilon$. We will manipulate this inequality to see what value of $N$ could produce a valid result.

\begin{center}
  \begin{tabular}{ccc}
    $|a_n - L| < \epsilon$ & $\iff$ \\
    $\ln(a_n - 0) < \ln(\epsilon)$ & $\iff$ & $[1]$ \\
    $\ln(2(\frac{1}{3})^{n-1}) < \ln(\epsilon)$ & $\iff$ \\
    $\ln(2) + (n-1)\ln(\frac{1}{3}) < \ln(\epsilon)$ & $\iff$ \\
    $n - 1 > \frac{\ln(\epsilon) - \ln(2)}{\ln(\frac{1}{3})}$ & $\iff$ & $[2]$ \\
    $n > 1 + \ln(\epsilon/2) \cdot \ln(1/3)^{-1}$

  \end{tabular} \\
  \begin{tabular}{l}
  $[1]$ $\epsilon$ and an abolute value are both positive so this is valid. \\
  $[2]$ The direction of the inequality flips because we divide by a negative number ($ln(1/3)$).
  \end{tabular}
\end{center}

We can see that the above equation is satisfiable, as any arbitrarily defined $\epsilon$ will produce a value for $n$ to exceed. In order for this to be complete we need to know that $a_n$ is strictly decreasing as well. This can be seen because for any arbitrary value $n > 0$, $2(\frac{1}{3})^{n-1} > 2(\frac{1}{3})^{n+1-1}$. Therefore the limit exists and is 0.

\subsection{Let $S_k := \sum\limits_{n=1}^{k} a_n$ be the sum of the first $k$ terms in the sequence.}
\subsubsection{Compute $S_1, S_2, S_3, S_4, S_5$.}
\begin{center}
\begin{tabular}{l}
  $S_1 = \sum_{n=1}^1 = 2(\frac{1}{3})^{n-1} = 2(\frac{1}{3})^{1-1} = 2$ \\
  $S_2 = \sum_{n=1}^2 = 2(\frac{1}{3})^{n-1} = 2 + \frac{2}{3} = \frac{8}{3}$ \\
  $S_3 = \sum_{n=1}^3 = 2(\frac{1}{3})^{n-1} = \frac{8}{3} + \frac{2}{9} = \frac{26}{9}$ \\
  $S_4 = \sum_{n=1}^3 = 2(\frac{1}{3})^{n-1} = \frac{26}{9} + \frac{2}{27} = \frac{80}{27}$ \\
  $S_5 = \sum_{n=1}^3 = 2(\frac{1}{3})^{n-1} = \frac{80}{27} + \frac{2}{81} = \frac{242}{81}$ \\
\end{tabular}
\end{center}

\subsubsection{Does the sequence $S_k$ converge or diverge as $k \rightarrow \infty$?}
As $k \rightarrow \infty$, sequence $S_k$ converges. We know from part (a) that limit of $a_n$ (L) is 0. We know that the summation of a series that converges to 0 will also converge to some value. This value does not have to be 0 though.

\subsection{Let $S := \sum\limits_{n=1}^{\infty} a_n$. Find $S - \frac{1}{3}S$. Call this expression ($\star$)}
$S$ here is equivalent to $\lim\limits_{k \rightarrow \infty} \sum\limits_{n=1}^k 2(\frac{1}{3})^{n-1}$. This summation converges to 3. Therefore $S$ is equal to 3.

Alternatively, symbolically this is
\begin{center}
  \begin{tabular}{cc}
    $\sum\limits_{n=1}^\infty 2(\frac{1}{3})^{n-1} - \frac{1}{3}\sum\limits_{n=1}^\infty 2(\frac{1}{3})^{n-1}$ & $\iff$ \\
    $\sum\limits_{n=1}^\infty 2(\frac{1}{3})^{n-1} - \frac{2}{3}(\frac{1}{3})^{n-1}$ & $\iff$ \\
    $\sum\limits_{n=1}^\infty \frac{4}{3}(\frac{1}{3})^{n-1}$ & $\iff$ \\
    $\frac{4}{3} \sum\limits_{n=1}^\infty (\frac{1}{3})^{n-1}$ & $\iff$ \\
  \end{tabular}
\end{center}

This means $S - \frac{1}{3}S = 2 = (\star)$ = $\frac{4}{3} \sum\limits_{n=1}^\infty (\frac{1}{3})^{n-1}$.

\subsection{Solve the equation $S - \frac{1}{3}S = (\star)$ for $S$. Explain what this solution tells you.}
Solving this equation algebraically produces $S = \frac{(\star)}{2 / 3} = 2\sum\limits_{n=1}^\infty (\frac{1}{3})^{n-1} = 3$.

This relationship is demonstrating how $\sum\limits_{n=1}^\infty ar^{n-1} - r\sum\limits_{n=1}^\infty ar^{n-1} = 1$. This means that in general $S = \frac{a}{1-r}$. In this problem we have $a=2, r = \frac{1}{3}$ so $S = 3$.

\newpage

\section{Question 5}
\begin{definition}
The $limit$ of a sequence $(a_n)$ as $n \rightarrow \infty$ exists and equals $L$ (denoted $\lim\limits_{n \rightarrow \infty} = L$) if and only if for any small number $\epsilon > 0$, there exists $N \in \mathbb{N}$ such that $|a_n - L| < \epsilon$ for all $n \geq N$.
\end{definition}
Define a sequence $a_n := 2(\frac{1}{3})^{n-1}$ for each $n \in \mathbb{Z}^+$

\subsection{What is the limit of $a_n$ as $n \rightarrow \infty$? Prove your answer using the definition above.}
We can demonstrate that the limit for the sequence $a_n$ is $0$ by demonstrating that there is some $N$ such that any $n \geq N$ the inequality $|a_n - L| < \epsilon$ for any arbitrary positive $\epsilon$. We will manipulate this inequality to see what value of $N$ could produce a valid result.

\begin{center}
  \begin{tabular}{ccc}
    $|a_n - L| < \epsilon$ & $\iff$ \\
    $\ln(a_n - 0) < \ln(\epsilon)$ & $\iff$ & $[1]$ \\
    $\ln(2(\frac{1}{3})^{n-1}) < \ln(\epsilon)$ & $\iff$ \\
    $\ln(2) + (n-1)\ln(\frac{1}{3}) < \ln(\epsilon)$ & $\iff$ \\
    $n - 1 > \frac{\ln(\epsilon) - \ln(2)}{\ln(\frac{1}{3})}$ & $\iff$ & $[2]$ \\
    $n > 1 + \ln(\epsilon/2) \cdot \ln(1/3)^{-1}$

  \end{tabular} \\
  \begin{tabular}{l}
  $[1]$ $\epsilon$ and an abolute value are both positive so this is valid. \\
  $[2]$ The direction of the inequality flips because we divide by a negative number ($ln(1/3)$).
  \end{tabular}
\end{center}

We can see that the above equation is satisfiable, as any arbitrarily defined $\epsilon$ will produce a value for $n$ to exceed. In order for this to be complete we need to know that $a_n$ is strictly decreasing as well. This can be seen because for any arbitrary value $n > 0$, $2(\frac{1}{3})^{n-1} > 2(\frac{1}{3})^{n+1-1}$. Therefore the limit exists and is 0.

\subsection{Let $S_k := \sum\limits_{n=1}^{k} a_n$ be the sum of the first $k$ terms in the sequence.}
\subsubsection{Compute $S_1, S_2, S_3, S_4, S_5$.}
\begin{center}
\begin{tabular}{l}
  $S_1 = \sum_{n=1}^1 = 2(\frac{1}{3})^{n-1} = 2(\frac{1}{3})^{1-1} = 2$ \\
  $S_2 = \sum_{n=1}^2 = 2(\frac{1}{3})^{n-1} = 2 + \frac{2}{3} = \frac{8}{3}$ \\
  $S_3 = \sum_{n=1}^3 = 2(\frac{1}{3})^{n-1} = \frac{8}{3} + \frac{2}{9} = \frac{26}{9}$ \\
  $S_4 = \sum_{n=1}^3 = 2(\frac{1}{3})^{n-1} = \frac{26}{9} + \frac{2}{27} = \frac{80}{27}$ \\
  $S_5 = \sum_{n=1}^3 = 2(\frac{1}{3})^{n-1} = \frac{80}{27} + \frac{2}{81} = \frac{242}{81}$ \\
\end{tabular}
\end{center}

\subsubsection{Does the sequence $S_k$ converge or diverge as $k \rightarrow \infty$?}
As $k \rightarrow \infty$, sequence $S_k$ converges. We know from part (a) that limit of $a_n$ (L) is 0. We know that the summation of a series that converges to 0 will also converge to some value. This value does not have to be 0 though.

\subsection{Let $S := \sum\limits_{n=1}^{\infty} a_n$. Find $S - \frac{1}{3}S$. Call this expression ($\star$)}
$S$ here is equivalent to $\lim\limits_{k \rightarrow \infty} \sum\limits_{n=1}^k 2(\frac{1}{3})^{n-1}$. This summation converges to 3. Therefore $S$ is equal to 3.

Alternatively, symbolically this is
\begin{center}
  \begin{tabular}{cc}
    $\sum\limits_{n=1}^\infty 2(\frac{1}{3})^{n-1} - \frac{1}{3}\sum\limits_{n=1}^\infty 2(\frac{1}{3})^{n-1}$ & $\iff$ \\
    $\sum\limits_{n=1}^\infty 2(\frac{1}{3})^{n-1} - \frac{2}{3}(\frac{1}{3})^{n-1}$ & $\iff$ \\
    $\sum\limits_{n=1}^\infty \frac{4}{3}(\frac{1}{3})^{n-1}$ & $\iff$ \\
    $\frac{4}{3} \sum\limits_{n=1}^\infty (\frac{1}{3})^{n-1}$ & $\iff$ \\
  \end{tabular}
\end{center}

This means $S - \frac{1}{3}S = 2 = (\star)$ = $\frac{4}{3} \sum\limits_{n=1}^\infty (\frac{1}{3})^{n-1}$.

\subsection{Solve the equation $S - \frac{1}{3}S = (\star)$ for $S$. Explain what this solution tells you.}
Solving this equation algebraically produces $S = \frac{(\star)}{2 / 3} = 2\sum\limits_{n=1}^\infty (\frac{1}{3})^{n-1} = 3$.

This relationship is demonstrating how $\sum\limits_{n=1}^\infty ar^{n-1} - r\sum\limits_{n=1}^\infty ar^{n-1} = 1$. This means that in general $S = \frac{a}{1-r}$. In this problem we have $a=2, r = \frac{1}{3}$ so $S = 3$.

\newpage

\section{Question 6}
\begin{definition}
The sequence $(a_n)$ diverges as $n \rightarrow \infty$ if and only if for any large number $M > 0$, there exists $N \in \mathbb{N}$ such that $|a(n)| > M$ for all $n \geq N$.
\end{definition}

Define a sequence $a_n := \frac{2n-1}{2}$ for each $n \in \mathbb{Z}^+$.

\subsection{Use induction to prove $\sum_{n=1}^k a_n = \frac{k^2}{2}$}
\subsubsection{Define a predicate $P(k)$.}
\begin{center}
$P(k) := \sum_{n=1}^k a_n = \frac{k^2}{2}$
\end{center}

\subsubsection{State what you are trying to show by quantifying your predicate.}
\begin{center}
$\forall k \in \mathbb{Z}^+ P(k)$
\end{center}

\subsubsection{Show $P(1)$ is true.}
\begin{center}
$P(1) = (\sum_{n=1}^1 a_n = \frac{1^2}{2})$ \\
$P(1) = (\frac{2 - 1}{2} = \frac{1}{2}$) \\
$P(1) = (\frac{1}{2} = \frac{1}{2})$ \\
$P(1)$ is True
\end{center}

\subsubsection{Prove that if $P(m)$ is true for some $k \in \mathbb{Z}^+$, then $P(m+1)$ is true.}
\begin{center}
\begin{tabular}{lll}
$P(m)$ is True. && Given \\

$P(m+1) := (\sum_{n=1}^{m+1} a_{n} = \frac{(m+1)^2}{2})$ & $\iff$ \\
$P(m+1) := (a_{m+1} + \sum_{n=1}^{m} a_{n} = \frac{m^2 + 2m + 1}{2})$ & $\iff$ \\
$P(m+1) := (a_{m+1} + \sum_{n=1}^{m} a_{n} = \frac{m^2}{2} + \frac{2m + 1}{2})$ & $\iff$ \\
$P(m+1) := (a_{m+1} = \frac{2m+1}{2})$ & $\iff$ & [1] \\
$P(m+1) := (\frac{2(m+1) - 1}{2} = \frac{2m + 1}{2})$ & $\iff$ \\
$P(m+1) := (\frac{2m + 1}{2} = \frac{2m + 1}{2})$ \\
$P(m+1)$ is True \\
\end{tabular} \\
$[1]$ Terms cancel because $P(m)$ is True.
\end{center}

This shows directly that we can remove a significant portion of the burden of the claim from $P(m+1)$ by knowing $P(m)$ is true.

\subsection{Prove that the sequence $S_k := \sum_{n=1}^k a_n$ diverges as $k \rightarrow \infty$}
The above proof has demonstrated  that this summation can be represented as the closed form formula $S_k := \frac{k^2}{2}$. This means we can demonstrate that $\frac{k^2}{2}$ diverges to signify that $\sum_{n=1}^k a_n$ diverges.

Consider some arbitrarily large number $M$ s.t. $M > 0$ Does there exist some $N \in \mathbb{N}$ such that $|S_k| > M$ for all $n \geq N$? If this $N$ exists it means that $S_N > M \equiv \frac{N^2}{2} > M$.

Algebraically solving for $N$ we find that $N > \sqrt{2M}$. This shows that there exists some $N$ that causes $S_N$ to be larger than M in all cases. Since $N \in \mathbb{N}$ we can see that $\frac{N^2}{2}$ is a strictly increasing function. This implies that if $S_N > M \rightarrow \forall n \geq N (S_n > M)$. This is the definition of a diverging series. Therefore $S_k$ diverges as $k \rightarrow \infty$.


\end{document}
